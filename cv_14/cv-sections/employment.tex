\cvsection{Employment}

\begin{cventries}
	
	
\cventry
{Lead Blockchain Researcher at Tomochain.com} % Degree
{Tomochain.com Pte. Ltd.} % Institution
{219 Trung Kinh, Hanoi, Vietnam} % Location
{May 2018 - Present} % Date(s)
{ % Description(s) bullet points
	\begin{cvitems}
		\item {I joint Tomochain in the middle of May as a Blockchain Researcher. My work is to collaborate with blockchain engineers to design the Tomochain public blockchain to achieve the goals of Tomochain including scalability and security while maintaining the decentralization of the system at an acceptable level. Specifically, we design our blockchain consensus algorithm, Proof-of-Stake Voting and are in way of implementing it. Also, we released our technical paper on the consensus and a series of in-depth technical articles focusing on the similarities and differences between Tomochain and other major blockchain infrastructures, including EOS, Cardano, Ethereum Casper, and Tendermint. I'm in charge of writing our technical documention, including papers and blogs, and analyzing technical strengths and limitations of our solution and the solutions of other blockchains such as Ethereum and EOS.}
		\item {I'm currently developing the solution to the blockchain challenging problem: blockchain sharding, in order to improve further the performance of Tomochain. The sharding design paper is currently released and I am integrating it into TomoChain's current architecture.}	
		\item {Plasma integration: My vision is that, if many DApps will be used by people, a single public blockchain will need both on-chain scaling solutions and off-chain solutions. Shading for on-chain scaling and Plasma for off-chain scaling are an idealized solution I'm developing. Each DApp can run on a Plasma side chain off the main chain and only sends settlement transactions to the main chain. }	
	\end{cvitems}
}	

\cventry
{Software Engineer/PhD Student at Laboratory of Model-Driven Engineering for Embedded Systems (LISE)} % Degree
{Laboratory of Model-Driven Engineering for Embedded Systems (LISE), CEA-LIST} % Institution
{CEA, Saclay, France} % Location
{Nov. 2014 - Nov. 2017 [anticipated]} % Date(s)
{ % Description(s) bullet points
	\begin{cvitems}
		\item {I have been working as a software engineering and PhD Student, under supervision of Dr. Ansgar RADERMACHER, Dr. S\'ebastien G\'ERARD, and Dr. Shuai LI, at LISE, which develops the Papyrus industrial modeling tool, especially the use of UML to design complex system software. My work involves: the use of UML for designing component-based architecture and event-driven behaviors for reactive and distributed systems in the context of the Papyrus Designer tool; the corresponding implementation of the UML-based design; the process of automatically translating the UML-based software design model into efficient executable code; and the process of automatically propagating modifications of the code back to the model. Technically, I have been working with several mainstream programming languages such as Java, C++, and C and have a deep understanding of these programming languages in order to generate code from models. I also extended the standard C++ programming language to support component-based developemnt for better managing complexity in code-centric development approaches, especially for reactive systems. An Arm-based Lego Car factory application is developed for demonstration.}		
	\end{cvitems}
}


\cventry
{Internship at Laboratory of Model-Driven Engineering for Embedded Systems (LISE)} % Degree
{Laboratory of Model-Driven Engineering for Embedded Systems (LISE), CEA-LIST} % Institution
{CEA, Saclay, France} % Location
{Apr. 2014 - Sept. 2014} % Date(s)
{ % Description(s) bullet points
	\begin{cvitems}
		\item {I worked as an intern undersupervison of Dr. Ansgar RADERMACHER in the context of applying component-based modeling and design to distributed system development. I used UML and a UML profile for designing interactions between distributed components in distributed systems. The created design is then automatically translated into code, which uses the ZeroMQ middleware to exchange data between distributed components.}		
	\end{cvitems}
}


\cventry
{Software Engineer for Computers and Embedded System} % Degree
{Toshiba Software Development Vietnam (TSDV)} % Institution
{Hanoi, Vietnam} % Location
{Jul. 2012 - Aug. 2013} % Date(s)
{ % Description(s) bullet points
	\begin{cvitems}
		\item {I worked as a software developer in three projects: (1) deal with requirement analysis, detailed design, implementation, unit test, integration test, and optimization at the programming level for software of an embedded system: Toshiba G2R protection relay. The development using C as the development language and Visual Studio as an integrated development environment (IDE) includes using IEC 60870-5-103 to control and communicate between smart electric device, and testing on real hardware and using debug tools to debug on hardware; (2) develop a desktop application to control electric devices by using a private protocol of Toshiba and the C\# language. The application polls the devices to receive data of recorded errors, and creates a chart for analysis of errors causes; and (3) use the Qt IDE and the C++ ffmpeg library for implementing a multimedia player application, which plays both audio and video files.}		
	\end{cvitems}
}

%\input{cv-sections/vietnaminternship}

\end{cventries}