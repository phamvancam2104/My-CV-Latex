\cvsection{Emploiement}

\begin{cventries}

\cventry
{Doctorant Étudiant au Laboratoire d'Ingénieurie Dirigée par les modèles pour les Systèmes Embarqués (LISE)} % Degree
{Laboratoire d'Ingénieurie Dirigée par les modèles pour les Systèmes Embarqués (LISE), CEA-LIST} % Institution
{CEA, Saclay, France} % Location
{Nov. 2014 - Présent} % Date(s)
{ % Description(s) bullet points
	\begin{cvitems}
		\item {Je travaille comme doctorant étudiant sous l'encadrement de Dr. Ansgar RADERMACHER, Dr. S\'ebastien G\'ERARD, et Dr. Shuai LI, au LISE, qui developpe l'outil de modélisation industriel, dans le domaine de l'MDE, notamment l'usage du langage de modélisation de UML et la synchronisation entre modèle-modèle et entre modèle-code généré. Mon travail implique l'utilisation de langages différents pour la transformation de modèles et la génération de code tels que QVT, ATL et Xtend, l'utilisation de l'UML pour modéliser des architectures basées sur des composants et des comportements dirigés par des événement pour les systèmes réagis dans le contexte de Papyrus - une extension de l'Eclipse Modeling Framework (EMF), et de techniques différents pour la synchronisation de modèle et code. En plus, Je travaille avec plusieurs langages de programmation dominants tels que Java, C++, et C et j'ai la compréhension profonde de ces langages afin de générer code à partir des modèles. J'ai également étendu le langage C++ pour supporter le desveloppement de systèmes réagis basés sur des composants.}		
	\end{cvitems}
}


\cventry
{Stage au Laboratoire d'Ingénieurie Dirigée par les modèles pour les Systèmes Embarqués (LISE)} % Degree
{Laboratoire d'Ingénieurie Dirigée par les modèles pour les Systèmes Embarqués (LISE), CEA-LIST} % Institution
{CEA, Saclay, France} % Location
{Avr. 2014 - Sept. 2014} % Date(s)
{ % Description(s) bullet points
	\begin{cvitems}
		\item {J'ai travaillé comme un stagiaire sous l'encadrement de Dr. Ansgar RADERMACHER dans le contexte of l'application de l'MDE and le modèle basé sur des composants FCM (component-based model) au développement de systèmes distribués, notamment la modélisation des composants d'intéraction entre des composants distribués basés sur le middleware ZeroMQ.}		
	\end{cvitems}
}


\cventry
{Ingénieur de logiciels pours des ordinateurs et des Systèmes Embarqués} % Degree
{Toshiba Software Development Vietnam (TSDV)} % Institution
{Hanoi, Vietnam} % Location
{Jul. 2012 - Aou. 2013} % Date(s)
{ % Description(s) bullet points
	\begin{cvitems}
		\item {J'ai travaillé comme un ingénieur de logiciels dans trois projets: (1) traiter l'analyse des exigences, le design detaillé, l'implémentation, les teste unitaires, les tests d'intégration, et optimisation au niveau de programmation pour le logiciels du système embarqué: Toshiba G2R protection relay. Le développement ustilisant C comme le langage de développement et Visual Studio comme l'environnement de développement intégré inclut l'utilisation du standard IEC 60870-5-103 pour contrôller et communiquer entre des appareils électriques intelligents, et le test sur un matériel réel et le test utilisant un outil de débogage; (2) développer une application pour contrôller les appareils électrique en utilisant un protocole privé de Toshiba et le langage de C\#. L'application sonde les appareils afin de recever des données des erreurs enregistrés, et créer un diagramme pour l'analyse des orgins des erreurs; et (3) utiliser le Qt framework et la bibliotheque C++ de ffmpeg pour l'implémentaion d'une application de lecture multimédia, qui joue les audios and les vidéo.}		
	\end{cvitems}
}

%\cventry
{Internship for developing games and applications for Android} % Degree
{Tamtay.vn Company} % Institution
{Hanoi, Vietnam} % Location
{Feb. 2012 - Apr. 2012} % Date(s)
{ % Description(s) bullet points
	\begin{cvitems}
		\item {I worked as an intern at the tamtay.vn company, that develops applications and games for the tamtay.vn social network. My responsibility is to investigate the development framework Android SDK for Android applications. I then collaborated with another intern to develop a simple multimedia player application, that downloads and plays music files and associated information transfered in JSON files. I was then responsible for investigating the AndEngine game engine to learn skills in game development.}		
	\end{cvitems}
}

\cventry
{Internship for investigating Google Cloud Platform} % Degree
{School of Communication and Information Technology - Hanoi University of Science and Technology} % Institution
{Hanoi, Vietnam} % Location
{Jul. 2011 - Aug. 2011} % Date(s)
{ % Description(s) bullet points
	\begin{cvitems}
		\item {I worked as an intern at the School of Communication and Information Technology. I investigated the \textit{Google App Engine} Google Cloud Platform for building scalable webs and mobile backends using Java, Eclipse IDE, TomCat, plugin of Google App Engine for Java.}		
	\end{cvitems}
}
\cventry
{Internship for developing games and applications for Android} % Degree
{Tamtay.vn Company} % Institution
{Hanoi, Vietnam} % Location
{Feb. 2012 - Apr. 2012} % Date(s)
{ % Description(s) bullet points
	\begin{cvitems}
		\item {I worked as an intern at the tamtay.vn company, that develops applications and games for the tamtay.vn social network. My responsibility is to investigate the development framework Android SDK for Android applications. I then collaborated with another intern to develop a simple multimedia play application, that downloads and plays music files and associated information transfered in JSON files. I was then responsible for investigating the AndEngine game engine to learn skills in game development.}		
	\end{cvitems}
}

\cventry
{Internship for investigating Google Cloud Platform} % Degree
{School of Communication and Information Technology - Hanoi University of Science and Technology} % Institution
{Hanoi, Vietnam} % Location
{Jul. 2011 - Aug. 2011} % Date(s)
{ % Description(s) bullet points
	\begin{cvitems}
		\item {I worked as an intern at the School of Communication and Information Technology. I investigated the \textit{Google App Engine} Google Cloud Platform for building scalable webs and mobile backends using Java, Eclipse IDE, TomCat, plugin of Google App Engine for Java.}		
	\end{cvitems}
}

\end{cventries}