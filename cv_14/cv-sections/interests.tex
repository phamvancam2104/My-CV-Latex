\cvsection{Topics of Interests}
\begin{cventries}

\cvinterest
{Blockchain infrastructure, scalability and applications}
{
	\begin{cvitems}
		\item{Blockchain infrastructure: I have had my deep knowledge of different blockchain infrastructures and understand the current problems of the mainstream infrastructure including Bitcoin, Ethereum, EOS, just to mention a few. My interest is devoted to blockchain scalability and security problems. Specifically, on on-chain scaling solutions, I have been developing my sharding solution to Tomochain and Ethereum as well and clearly understanding the sharding of some emerging platforms such as Zilliqa and Quarkchain. I have released a sharding architecture for TomoChain and currently is integrating it into TomoChain. The sharding design has some uniqueness such as an \textit{incentive-driven verification game} or several cross-shard transaction schemes. On off-chain scaling solutions, I have developed deep knowledge on Plasma and State Channel, as well as Lightning Network of Bitcoin. My vision is to have a main scalable blockchain on which each DApp can run on a Plasma side chain, which can significantly increase scalability while still maintaining security properties.
		As a Plasma enthusiast, I look forward to collaborating with researchers and developers who have special interest in Plasma.}
		\item{Ethereum smart contract: I have also developed my deep knowledge in Ethereum Virtual Machine, EVM opcodes and currently been involving smart contract security best practices and security formal verification tools such as Mythril and Securify. Hacking smart contracts is also my favorite game}
		%\item{Blockchain applications: I'm much interested in how the blockchain technologies can be applied to different sectors, including supply chain management, fintech (i.e. payment channel using state channels), health-care and data exchange.}
	\end{cvitems}
}

\cvinterest
{Software programming language, software architecture design and implementation}
{
	\begin{cvitems}
		\item{As a PhD student and a practitioner in software engineering, I am passionate in programming, especially the use of mainstream programming languages such as C/C++, Java, and Python for development of software applications (e.g. reactive embedded system applications) for solving everyday life problems. I'm also interested in knowing practices how to be productive and qualitative at programming.}
		\item{A major part of my thesis work is about extending, engineering, and transforming software programming language. Especially, I'm interested in extending current programming languages to assist developers to be productive and qualitative in software development.}
		\item{During PhD time, I have learned about application of event-driven architecture to asynchronous systems such as distributed systems. I'm interested in using programming model and network protocols to develop distributed and/or embedded applications, which will involve network technologies such as communication protocols or routing algorithms as well as the combination of peer-to-peer and client-server model as in my undergraduate at HUST.}
	\end{cvitems}
}


\cvinterest
{Research and Development of Embedded and Distributed Systems}
{\begin{cvitems}
		\item{In my thesis, I work in the context of development of embedded software systems, in which I use UML and design patterns for designing such systems (or software in general). I'm therefore interested in diving into this domain as deep as possible. Furthermore, I want to combine my knowledge about the network technologies that I have collected to produce distributed embedded software systems. The latter, in my thought, are very suitable to the development (design and coding) of IoT applications, in which devices collect, process, and exchange data with each other, and which might be combined with machine learning field. Using UML, from my point of view, would be very helpful for the complexity management task of such systems.}
\end{cvitems}}

%\cvinterest
%{Machine learning}
%{\begin{cvitems}
	%	\item{The field of machine learning and its application have been attracting many practitioners including myself.
%During my master education, I was taught about machine learning (modules data fusion and statistic learning and neural network). 
%Since there, I have been keeping my passion and learning practically, especially using Python, the use of different learning models for understanding data and prediction.}
%\end{cvitems}
%}

\cvinterest
{Model-Based Software Engineering}
{
	\begin{cvitems}
	\item{The main research interest of my PhD thesis are methodologies for synchronization of code and model specified in UML-based component-based modeling and UML state machines, and studying different techniques for model transformation such as QVT, ATL, Triple-Graph Grammar (TGG), and Change-driven transformation, for model synchronization such as QVT-Relation and TGG.	
I'm also involved in researching approaches for a mapping of a UML model containing the ALF code, and programming language code.
This mapping will enable a synchronization of UML model with ALF and programming code, and eventually provide synchronization of a platform-independent model with code.
Furthermore, I'm also interested in approaches for simulating a system from its models and generating fully operational code from models, especially UML models, which might contain ALF code, approaches for optimizing software system at the model level (e.g. optimization ALF code or UML state machines), and if possibly approaches for model interpretation and compilation.
The final goal is harmonization of software programming and MDE practices.}
\end{cvitems}}


\cvinterest
{Leisure}
{Football watching, tourism, book reading (especially historical + technical books, or sometimes literature), chatting}

\begin{comment}
\cvinterest
{Model-Driven Engineering with Large Models}
{
	\begin{cvitems}
		\item {I have been working with Papyrus UML models during my thesis for generating real case-study embedded systems, e.g. 12000 lines of code generated for the Lego Car factory case study. When models become large, the processing of the models, e.g. querying and loading for model transformations, becomes very slow, which might harms the adoption of modeling techniques to industrial development. For this, I'm very interested in studying techniques for speeding model manipulations up such as incremental model query with IncQuery or the PrefetchML prefetching and caching language for caching models.}
	\end{cvitems}
}
\end{comment}


\end{cventries}

