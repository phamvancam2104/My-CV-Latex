\cvsection{Topics of Interests}
\begin{cventries}

\cvinterest
{Model-Driven Software Engineering}
{
	\begin{cvitems}
	\item{The main research interest of my PhD thesis are methodologies for synchronization of code and model specified in UML-based component-based modeling and UML state machines, and studying different techniques for model transformation such as QVT, ATL, Triple-Graph Grammar (TGG), and Change-driven transformation, for model synchronization such as QVT-Relation and TGG.	
I'm also involved in researching approaches for a mapping of a UML model containing the ALF code, and programming language code.
This mapping will enable a synchronization of UML model with ALF and programming code, and eventually provide synchronization of a platform-independent model with code.
Furthermore, I'm also interested in approaches for simulating a system from its models and generating fully operational code from models, especially UML models, which might contain ALF code, approaches for optimizing software system at the model level (e.g. optimization ALF code or UML state machines), and if possibly approaches for model interpretation and compilation.
The final goal is harmonization of software programming and MDE practices.}
\end{cvitems}}

\cvinterest
{Software programming}
{
	\begin{cvitems}
		\item{As a PhD student and a practitioner in software engineering, I am passionate in programming, especially the use of mainstream programming languages such as C/C++ and Java for development of software applications (e.g. embedded system applications) for solving everyday life problems. I'm also interested in knowing practices how to be productive and qualitative in programming.}
	\end{cvitems}
}


\cvinterest
{Model-Driven Engineering with Large Models}
{
	\begin{cvitems}
		\item {I have been working with Papyrus UML models during my thesis for generating real case-study embedded systems, e.g. 12000 lines of code generated for the Lego Car factory case study. When models become large, the processing of the models, e.g. querying and loading for model transformations, becomes very slow, which might harms the adoption of modeling techniques to industrial development. For this, I'm very interested in studying techniques for speeding model manipulations up such as incremental model query with IncQuery or the PrefetchML prefetching and caching language for caching models.}
	\end{cvitems}
}

\cvinterest
{Application to Modeling and Development of Embedded and Distributed Systems}
{\begin{cvitems}
		\item{In my thesis, I work in the context of the Papyrus Designer - an extension of the Papyrus modeling tool.
Papyrus Designer provides component-based modeling and code generation for distributed embedded systems, using the concepts of interaction components to model remote interactions between distributed components.
I'm interested in studying methodologies for applying Papyrus Designer with its UML profile to distributed systems, especially component-based distributed reactive systems.}
\end{cvitems}}

\cvinterest
{State Machine-Based Language Engineering and Transformation}
{\begin{cvitems}
		\item{A major part of my thesis work is about extending, engineering, and transforming software programming language.
Especially, I'm interested in extending current programming languages to support a state machine-based event-driven model for developing reactive systems.}
\end{cvitems}}


\end{cventries}

