%----------------------------------------------------------------------------------------
%	SECTION TITLE
%----------------------------------------------------------------------------------------

\cvsection{Education}

%----------------------------------------------------------------------------------------
%	SECTION CONTENT
%----------------------------------------------------------------------------------------

\begin{cventries}

%------------------------------------------------

\cventry
{Thesis Title: Methothodologies for model-code synchronization for reactive system development} % Degree
{Software Engineer/PhD at LISE (Laboratory of Model-Driven Engineering for Embedded Systems)} % Institution
{Saclay, France} % Location
{Nov. 2014 - Jan. 2018} % Date(s)
{ % Description(s) bullet points
	\begin{cvitems}
		\item {\textbf{Working place}: Laboratory of Model-Driven Engineering for Embedded Systems (LISE), CEA-List, Saclay, France.}	
		\item {\textbf{Brief Synopsis}: I have been working as a software engineering as well as a PhD student in the Laboratoire d'Ingénierie dirigée par les modèles pour les systèmes embarqués in the domain design and implementation of reactive application in the context of Model-Based Software Engineering (MBSE) and component-based architecture design. 
		MBSE focuses on using abstract diagram-based modeling languages such as UML to design complex system architecture. 
		I develop a framework, which allows to (1) use UML to design complex system software and automatically generate C/C++ code of the high-level application behavior; (2) develop algorithmic/computational code for applications; and (3) propagate the modifications in code back to the UML design to update the design and to keep it consistent with the code. 
		Code productivity is significantly gained by the ability to automatically produce code from design. Furthermore, C/C++ code is efficient and many bugs in code are eliminated because of the use of automatic code generation. That is to say that, the approach benefits advantages of both of MBSE such as complexity management, productivity  and communication by UML, and traditional software programming, which is one of the most important tasks during software development with many strong and widely used programming languages such as Java and C++.  
		%My thesis is to propose an approach for combining both of the modeling and programming practice to benefit their advantages. 
		%Specifically, I propose a synchronization approach, which keeps UML-based software architecture model and object-oriented programming code such as C++ and Java consistent in case there are modifications of the model and the code. 
		%The software structure is designed based on component-based engineering, which enables decoupling between software components and re-usability of components. 
		%The software behavior is described by using state machines, which are appropriate to manage the discrete event-driven reactive system behaviors.
	} 
	\end{cvitems}
}

\cventry
{Master in Embedded Systems and Signal Processing} % Degree
{U-PSUD(University of Paris-Sud)} % Institution
{Orsay, France} % Location
{Sept. 2013 - Sept. 2014} % Date(s)
{ % Description(s) bullet points
	\begin{cvitems}
		\item {Program: Embedded Systems and Signal Processing (SETI) cooperated by University of Paris Sud, ENS Cachan, INSTN and ENSTA ParisTech, France}		
		\item {Scholarship Student of International Relationships of University of Paris-Sud}
		\item {Modules included: Real-time Control Numeric Systems, Complex Embedded System Design, Algorithm-Architecture Ad-equation, Network and Quality of Services, Multimedia Data Compression, Data Fusion, Statistic Learning and Neural Network, and Initiation to Research.}
	\end{cvitems}
}


\cventry
{Engineer in Information System and Communication} % Degree
{HUST(Hanoi University of Science and Technology)} % Institution
{Hanoi, Vietnam} % Location
{Sept. 2007 - Jul. 2012} % Date(s)
{ % Description(s) bullet points
\begin{cvitems}
\item {Program: Programme de Formation d'Ingenieur d'Excellence au Vietnam (PFIEV) - Programming of Training for Excellent Engineers in Vietnam}
\item {Final project: Understanding pros and cons of the client-server and peer-to-peer models, and combining these two models for developing distributed applications.}
\end{cvitems}
}

%------------------------------------------------

\end{cventries}