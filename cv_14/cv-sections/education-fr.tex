%----------------------------------------------------------------------------------------
%	SECTION TITLE
%----------------------------------------------------------------------------------------

\cvsection{Formation}

%----------------------------------------------------------------------------------------
%	SECTION CONTENT
%----------------------------------------------------------------------------------------

\begin{cventries}

%------------------------------------------------

\cventry
{Thèse: Methothodologies for model-code synchronization for reactive system development} % Dégrée
{Doctorant Étudiant au LISE (Laboratoire d'Ingénieurie Dirigée par les modèles pour les Systèmes Embarqués)} % Institution
{Saclay, France} % Location
{Nov. 2014 - Présent} % Date(s)
{ % Description(s) bullet points
	\begin{cvitems}
		\item {\textbf{Place de travail}: Laboratoire d'Ingénieurie Dirigée par les modèles pour les Systèmes Embarqués (LISE), CEA-List, Saclay, France.}	
		\item {\textbf{Bref Résumé de la Recherche}: L'utilisation de Model-Driven Engineering (MDE) dans l'industrie augmente. Cependant, d'une part, son adoption est encore faible en comparaison avec l'usage des langages de programmation dans l'ingénierie de logiciels. D'autre part, le code entièrement opérationnel est à peine généré à partir des modèles écrits dans des langages de modélisation graphiques, en utilisant les approches de l'MDE actuelles. Cette thèse proposes une approche for synchroniser le code et les modèles spécifiés en utilisant la modélisation basée sur des composants et les UML state machines (machines à états) pour le développement de systèmes embarqués réagis. 
		L'objective est triple: (1) améliorer la flexibilité de l'MDE en permettant de modifier un modèle et son code généré; (2) harmoniser l'MDE avec des pratiques de programmation traditionnelles; et (3) favoriser la collaboration entre des utilisateurs de l'MDE et développeurs traditionnels de logiciels.} 	
		\item {\textbf{L'intérêt de la recherche}: Ma recherche actuel focalise autour de l'MDE et ses applications, la transformation et la synchronisation de modèles, et l'ingénierie de langages de logiciels}
	\end{cvitems}
}

\cventry
{Master 2 en Systèmes Embarqués et Traitement du Signal (SETI)} % Degree
{U-PSUD (Université de Paris-Sud)} % Institution
{Orsay, France} % Location
{Sept. 2013 - Sept. 2014} % Date(s)
{ % Description(s) bullet points
	\begin{cvitems}
		\item {Programme: Systèmes Embarqués et Traitement du Signal coopéré par l'Université de Paris Sud, ENS Cachan, INSTN CEA et ENSTA ParisTech, France}		
		\item {Étudiant en bourse de les Relations Internationales de l'Université de Paris-Sud}
		\item {Modules inclus: Systèmes Numériques de Commandes en temps réel, Conception de Systèmes Embarqués Complexes, Adéquation de Algorithme-Architecture, Réseaux et Qualité de Services, Compression de Données Multimédia, Fusion de Données, Apprentissage et Réseaux de neurones, et Initiation à la Recherche.}
	\end{cvitems}
}


\cventry
{Ingénieur en Systèmes de l'Information et de la Communication} % Degree
{IPH (Institut Polytechnique de Hanoi)} % Institution
{Hanoi, Vietnam} % Location
{Sept. 2007 - Jul. 2012} % Date(s)
{ % Description(s) bullet points
\begin{cvitems}
\item {Programme: Programme de Formation d'Ingenieur d'Excellence au Vietnam (PFIEV)}
\item {Projet final: Comprendre les avantages et les inconvénients du modèle client-serveur et du modèle peer-to-peer, et combiner les deux modèles pour développer les applications distribuées.}
\end{cvitems}
}

%------------------------------------------------

\end{cventries}